\documentclass[12pt]{report}

\usepackage[utf8]{inputenc}

\usepackage{biblatex}
\usepackage{amsmath,amssymb,amsthm}
\usepackage[croatian]{babel}
\usepackage{csquotes}
\MakeOuterQuote{"}

\usepackage{thmtools}
\declaretheorem{teorem}
\declaretheorem[sibling=teorem]{korolar}
\declaretheorem[name=Činjenica,sibling=teorem,qed=\textup{EOČ}]{cinjenica}
\declaretheorem[style=definition,sibling=teorem,qed=$\vartriangleleft$]{definicija}
\declaretheorem[style=remark,sibling=teorem]{napomena}

\usepackage{biblatex}
\addbibresource{bibliografija.bib}

\title{Ethereum: Smart contracts}
\author{Luka Seničić}
\date{\today}

\begin{document}

\maketitle

\tableofcontents

\chapter{Uvod}
\section{Motivacija}
digitalne valute pokusavaju rjesit problem automated payment sistema

\section{Povijest}
Prva ideju i sam izraz \emph{smart contract} objavio je Nick Szabo 1994. godine\cite{smart_contract_idea}. U tom članku on smart contract predstavlja kao:
\begin{definicija}
Smart contract je digitalni transakcijski protokol koji izvršava odredbe ugovora. Ciljevi dizajna smart contracta su da zadovoljava uobičajene zahtjeve u ugovorima (načine plaćanjam, anonimnost itd.), minimiziraju potrebu za povjerljivom trećom osobom.
\end{definicija}
Kao primjer nanjbolji smart contracta daje DigitalCash, koji se smatra prvom digitalnom valutom i pojavio se 1982. godine\cite{digi_cash}. Szabo vrlo ambiciozno širi svoju ideju na razne načine. Koristi smart contract za izvršavanje kompleksnih načina plaćanja uz malu naknadu i jednostavnu izvedbu. Čak daje primjer i implementacije smart contracta u fizički objekt. auto....


\section{Thesis outline}

\chapter{Blockchain}

\section{Kriptografija}

\section{Uvod u Blockchain}

\section{Bitcoin}

\chapter{Ethereum}

pokupljeno sa \cite{ethdocs}

\section{Decentralized application platform}

\section{Accounti}

\section{Ether \& Gas}

\section{Transakcije}

\section{Protocol \& mining}

\chapter{Smart Contracts}

\section{Solidity \& EVM}

pokupljeno sa \cite{soliditydocs}

\section{Komunikacija}

\section{Sigurnost}

\chapter{Primjer:}

\chapter{Zaključak}

\printbibliography--

\end{document}
